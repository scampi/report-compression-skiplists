\chapter{SIREn Query Parser}{
The query parser is used in SIREn in order to allow people to interact with its
index. However we must keep in mind that Sindice, which uses SIREn for
searching and browsing purpose, is aimed to be used as a web service by machine
as well. In this section, we present a query parser that matches triple
patterns in datasets before describing a parser that matches entities.
}
\label{chap:siren-extension}
\section{NTriple Query Parser}
\label{sec:N3-qparser}

As with the SPARQL language, the NTriple query parser aims to find sub-graph
within datasets that best match a given set of triple patterns. Triple patterns
are linked together thanks to Boolean operators, i.e. AND, OR, NOT. To express
such patterns, SIREn uses a bottom-up grammar, and is built as a plugin to the
Lucene\footnote{Lucene: \url{http://lucene.apache.org/}} query parser.
\begin{description}
  \item[Literals] any string enclosed between simple quotes are taken as a
  literal.
  \item[Literal Pattern] any string enclosed between double quotes. It
  describes the pattern that has to appear within the object of the returned
  triples for the query, thanks to Boolean operations.
  \item[URIs] any string enclosed between the characters '$<$' \whs '$>$'.
\end{description}
Any part of a triple pattern, to a maximum of two per triple, can be left
unknown thanks to the wildcard character '*'. Any term of the query can be
flagged in order to indicate if the term has to, must not or may appear within
the triples in the results by using respectively the characters '+', '-', ''
(i.e no character). Considering URIs are tokenism when indexing (e.g.
\url{http://sindice.com/} is tokenism into ``HTTP'', ``Sindice'' and ``com''),
the following query
\begin{quotation}
$<$rings$>$ $<$author$>$ ``j.r.r.\_tolkien OR tolkien'' .
\end{quotation}
matches the RDF graph of the Figure~\ref{fig:rdf-graph}. The object of this
triple pattern is a literal pattern, which aims to widen the search to several
writings of the author's name. Since SIREn is an Information Retrieval search
engine, it processes the query as explained in Chapter~\ref{sec:IR} with set
theory operations (i.e. conjunction, disjunction and exclusion).

During indexing, it is possible to store into a specific field any text. For
instance, one can have three different fields such as ``title'', ``author'' and
``abstract'' for the entities in the Figure~\ref{fig:entities}. This way the
abstract of the entity \emph{\_:bnode1} would fall into its field, allowing for
more precise search. A field is identified with the special Lucene character
semi-colon, ':'. For instance the term ``hobbit'' within the entity \_:bnode1
can be refer ed as \emph{abstract{\bfseries :}hobbit}. 

\subsection{Added Features}

The SIREn query parser has been extended to support commonly used feature in
RDF and also to be more expressive. In this section three added features are
presented.

\paragraph{URI Pattern}

As with literal patterns, this extension allows to use Boolean operators within
an URI. It is then possible for example to search for RDF graphs where some
URI(s) contains both the terms \emph{lord} and \emph{rings} by entering the
query \emph{$<$lord AND rings$>$ * * .}$\;$.

\paragraph{Qualified Name - QName}

As explained in the introduction to Information Retrieval in
Chapter~\ref{sec:IR}, a QName is a term that is used to abbreviate an URI.
Whiting an URI, it is possible to use a QName by appending the term to the
semi-colon character. The QName is replaced at query processing time by the
correct namespace thanks to a file containing the mapping between a QName and
its expansion.
This expansion is performed on any string that matches a QName format only
within an URI, and in that case leaves the term unchanged if no mapping key has
been found. Restricting the substitution to URIs allows to QNames-like strings
within literals.

A QName can possess several expansion, e.g. the web
site\footnote{QNames namespace lookup for developers:
\url{http://prefix.cc/}} reports that the QName \emph{dbpedia} is mapped to
three possible expansion
\begin{enumerate}
  \item \url{http://dbpedia.org/resource/}
  \item \url{http://dbpedia.org/dbprop/}
  \item \url{http://dbpedia.org/property/}
\end{enumerate}
In order to deal with multi-valued QNames, all possible expansion replace their
QName, linked together with the Boolean operation OR. For instance the query
\begin{quotation}
* $<$\textcolor{red}{dbpedia}:\textcolor{blue}{title}$>$ * .
\end{quotation}
is expanded to
\begin{quotation}
\resizebox{\linewidth}{!}{%
* $<$\textcolor{red}{\url{http://dbpedia.org/resource/}}
\textcolor{blue}{title} OR
\textcolor{red}{\url{http://dbpedia.org/dbprop/}}\textcolor{blue}{title} OR
\textcolor{red}{\url{http://dbpedia.org/property/}}\textcolor{blue}{title}$>$ * .
}%
\end{quotation}
This expansion is rendered possible thanks to the URI pattern extension of an
URI presented previously. Thus it allows to perform more generic search and to
abstract from the actual schema definition the RDF graphs possess.

\paragraph{Multi-field search}

SIREn is based on Lucene, which data representation model uses fields. Fields
allows to organize logically the information. For example we can split the
documents from the collection according to their dataset, so that all documents
from a same dataset are within the same field.

However spitting information into fields implies that we lose a global view of
the data. To cope with this drawback, we have implemented an extension to the
query parser that search documents across fields, enabling the retrieval of
triple patterns regardless of the field it actually occurs. Moreover depending
on the reliability or the importance that is given to field, it is possible to
affect a \emph{weight} value to that field.

As an example, below is shown a RDF graph with two triples taken from two
different datasets: ``dataset1'' and ``dataset2''
\begin{quotation}
dataset1: $<$http://s$>$ $<$http://p1$>$ ``literal'' .

dataset2: $<$http://s$>$ $<$http://p2$>$ $<$http://o2$>$ .
\end{quotation}
While the query with two triple patterns
\begin{quotation}
($<$http://s$>$ * 'literal' .) AND ($<$http://s$>$ *
$<$\textcolor{blue}{http://o2}$>$ .)
\end{quotation}
matches the previous graph thanks to the implicit link, i.e., the first pattern
returns the triple from dataset1 and the second one from dataset2, the query
\begin{quotation}
($<$http://s$>$ * 'literal' .) AND ($<$http://s$>$ *
$<$\textcolor{blue}{http://o1}$>$ .)
\end{quotation}
does not, since the second pattern cannot be found.

\section{Entity Query Parser}
\label{sec:ent-qparser}

With the semantic web representation model such as RDF, we can describe people,
products, countries or any other entities. When searching for information we
often want results that are relevant for a specific entity, any other results
polluting the information retrieved. As a first step, SIREn is indexing
entities with an entity-centric rather than a document-centric model. The
second step is to build a parser that handles queries for an entity.

The entity query parser is being developed which goal is to matcher patterns for
an entity. For example we will able to formulate the query ``Find all the
people within DERI that are from France''. To better represent relations
between an entity and other entities or objects, we use a one-to-many
representation model called the \emph{tabular} format. The
Table~\ref{tab:tabular-format} depicts triples in a document-centric
representation on the left, and its tabular view on the right. With the tabular
format we define as a field the predicate (i.e., attribute) and store within
that field all the related objects. To answer the previous example query, we
can have two fields in the entity description of DERI, the countries and the
people fields.

For such queries to be efficient the schema of the data has to be known. As
such this parser is intended to be use over specific datasets, such as the
intranet of a company.

\begin{table}
\ra{0.3}
\centering
\begin{tabular}{lllc@{\hs}ll}
\toprule
\multicolumn{3}{c}{Document-centric} & \phantom{a} & \multicolumn{2}{c}{Tabular
format} \\
$<$ s $>$ & $<$ p1 $>$ & $<$ o1 $>\;.$ & \phantom{a} &
\multirow{2}{*}{$<$ o1 $> \; <$ o2 \; o3 $>$} \\
$<$ s $>$ & $<$ p2 $>$ & $<$ o2 $> \; <$ o3 $>\;.$ & \phantom{a} & & \\
\bottomrule
\end{tabular}
\caption{The tabular format data representation.}
\label{tab:tabular-format}
\end{table}


\chapter{Conclusion}{
In the recent years, the amount of semantic data has considerable increased.
More and more web sites are exporting their data using RDF, as the power of the
semantic web is increasingly attractive: to be able to efficiently search
across multiple data sources specific information, and to retrieve relevant
results hereafter. The semantic Web is a challenging and expanding research
domain. The Web as we know it is undergoing a radical change, and the semantic
web is contributing a lot to it.

People are publishing RDF data following the best practices of Linked Data. The
Linking Open Data Cloud is a huge gathering of inter-connected semantic data
sources. Applications can use this linked data and provide a concrete benefit
to the way we use the Web. \emph{Sig.ma} is a use case for a mashup application
that shows the power of the semantic. Its purpose is to search across many
data sources, and to provide information organized around \emph{entities},
e.g., a product, a person or any other concept. To be able to efficiently use
this collection of knowledge of we scale so that applications like Sig.ma are
viable is an important and challenging matter. Sig.ma is built on
\emph{Sindice}, a web service that provides search and retrieval capabilities
over semantic data. The web service uses \emph{SIREn} at its core, an
Information Retrieval search engine, to query an \emph{information need} and
to retrieve \emph{relevant} documents.

In this report we presented data structures that are commonly used in
Information Retrieval search engines. We also discussed about some of the
focusing points for optimizing these structures in order to have more efficient
and scalable IR search engines like SIREn.
We presented AFOR, a compression method that provides both fast compression
(increases index updates) and decompression (increases query throughput) speed,
and yet with a high compression ratio. We proposed SkipBlock, a novel
self-indexing model where some configurations carefully chosen provide faster
random lookups from an inverted list and a more compact structure than the
original Skip List model.}
\label{chap:conclusion}

\section{Summary of the Report}

The flow of this report reflects the flow of a research work by
\begin{inparaenum}[(1)]
\item describing a problem and why current related work do not answer the
requirements;
\item proposing a solution to the problem; and
\item proving the previous claims by performing comparative benchmarks.
\end{inparaenum}
Following this pattern, we proposed two novel structures, where both have the
common goal to reduce the amount of data read in order to increase the IO
throughput. 
\begin{description}
\item[Compression Technique] compression techniques do not only aim at reducing
the storage space, but also at increasing the IO access time. Reading/writing
less data from/to disk with a high performance algorithm reduce the wasted time
on IO access. Thus the performance of operations that directly depends on some
data to process is improved. We proposed \emph{AFOR}, a new compression class
that can increase query throughput compared to other state of the art
algorithms thanks to a more ``close-to-data'' compression.
\item[Self-Indexing Technique] query processing returns relevant information by
applying some operations on the inverted lists. However not all the data that
inverted lists have is necessary, thus reading or decoding such data is a
wasted time. Self-indexing is a technique that allows to skip over portions of
the inverted lists that are unnecessary for the processing of a query. We
proposed a new self-indexing model, called \emph{SkipBlock}, that aims to
improve the original model Skip List by taking into consideration the
compression algorithm used on the inverted lists. We will also present at
\emph{The $33^{rd}$ European Conference on Information Retrieval}
(ECIR)\footnote{ECIR: \url{http://www.ecir2011.dcu.ie/}} the paper which
introduced the model (Appendix~\ref{app:SkipBlock-paper}).
\end{description}

\section{Future Work}

The Information Retrieval domain for Semantic Data covers not only what has
been presented in this report but a wider area of problems. In the coming
months, I will stay within DERI and work on different subjects. In this section
I list a number of the possible future work.
\begin{itemize}
  \item Finalize the AFOR implementation to make it into a production ready
  state.
  \item Continue the research on SkipBlock, which will consists in optimizing
  search strategies and finding more adapted ones.
  \item Implement a novel SIREn index structure that will improve query
  processing performance.
  \item Start researching on dynamic query processing.
\end{itemize}

\section{Personal Benefits}

My internship at DERI has been a source for new knowledge in many areas. I was
able to deepen not only my skills in computer programming but also my
scientific knowledge.
\begin{description}
\item[Computer Skills] Thanks to this internship I was able to improve my
skills on different programming languages: in JAVA since our project SIREn
uses it, in script shell such as bash or Ruby when writing benchmarks automating
scripts, in a text stream editor like \emph{Sed}, very convenient for automate
operations on large files such as logging or benchmarking results files.
\\*[\parskip]
Because SIREn is a system built to be highly efficient and scalable over
millions of entities descriptions, any code that will be used within has to be
well written. The engineer must take care of the memory and the CPU consumption
so that the best performance is reached. Concerning JAVA there are many classes
available to help the developer, for instance it is better to use the class
\texttt{StringBuilder}\footnote{\texttt{StringBuilder}:
\url{http://tinyurl.com/3xbkvw6}} when operating of very large strings than to
use the \texttt{String} type.
\\*[\parskip]
At last it is important to comment the code written, not only for people using
it later on but also for ourselves since it permits to know its structure or
what are the possible optimizations. As part of commenting the code, writing
meaningful descriptions in SVN logs helps to keep track of what was done and
the reason of some changes.
\item[Engineering Skills] For the last months of my internship I was given a
project (the SkipBlock model) to work on alone. This experience showed me the
different points to take care of when managing projects, such as coordinating
the development with some deadline. Also when implementing a solution, there
are sometimes a difference between the model and the real results of the
implementations. This leads to the necessity of taking decisions in order to
understand why it is so and to be able to explain clearly the reasons. Moreover
it is frequent when implementing under a deadline pressure that the code isn't
optimized. In a short term this is not a problem, but in a long term it becomes
one as a \emph{technical debt}\footnote{Technical Debt:
\url{http://www.martinfowler.com/bliki/TechnicalDebt.html}}, since a messy code
will end up in re-factoring.
\item[Research Skills] DERI made me aware of the challenges that we can expect
from the research environment, such as a research dependent implementations
which change a project flow, since time constraints cannot be put on tasks
because the expecting difficulty and problems are still unknown. Moreover
working on the SIREn project allowed me to take part into the publication
process of scientific papers. These points as well as the scientific domain of
DERI (Semantic Web) and of the project (IR plus highly efficient structures)
gave me the desire to keep on working in the DI2 team for the following year.
\item[Scientific Knowledge] Thanks to this internship I was able to gain more
knowledge on the interesting Information Retrieval domain. Moreover the
internship made me aware of the Semantic Web infrastructure and of all the
possibilities it provides.
\item[Human Skills] A project cannot be successful unless the communication
between members is clear and efficient. A good communication flow allows
members to know the big picture of the research, and what should be working on
each individuals. DERI is a multi-cultural research institute, with
people form all around the world. This working environment was a good basis for
improving my English.
\end{description}