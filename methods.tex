\chapter{Inverted Lists Compression For SIREn}{%
The SIREn indexing model generates a specific distribution of values. Since
each stream of values are local to an other (i.e attributes identifiers are
local to an entity, values identifiers are local to an attribute, and positions
are local to a value), a new compression algorithm that is better adapted to
the SIREn use case was researched.

The performance aspects looked after are
\begin{inparaenum}[(1)]
\item the compression
\item the decompression and
\item the compression ratio.
\end{inparaenum}
The decompression speed is important when processing queries because it means
that less time is to be spent on IO operations. The compression ratio is
important as per the storage space, but not only. As previously, less data
read from disk means less time wasted on IO operations. The compression speed
aspect does not matter when processing queries, but takes an important place
when updating the index. Indeed SIREn possesses an incremental index update
policy, thus compression algorithm showing good decompression speed will allow
indexes to be updated faster than an algorithm with slower compression
algorithm.

In this chapter, two of the compression techniques researched on are presented,
with \emph{RiceFOR} first, an algorithm that fuses together the Rice and FOR
algorithms, and the \emph{Adaptive Frame Of Reference} (AFOR), a new algorithm
based on FOR, which goal is to decrease the FOR sensibility to outliers.
}%
\label{chap:cmp-methods}
\section{RiceFOR}{
Rice is an algorithm that shows compression ratio
performance~\cite{zhang:2008:www} close to the interpolative coding, which is
an algorithm that provides the best compression ratio. However its complexity
is so high that it is used only as a comparison basis between algorithms.
RiceFOR is an attempt to take the best from both Rice and FOR algorithms: the
compression ratio of Rice and the highly CPU optimised FOR execution flow.
}

\subsection{Algorithm}

In the one hand, Rice computes the average of a block of values. Thanks to a
parameter $b$ being the power of 2 closest to this average, it computes the
Euclidean division of each values by b. The remainder is compressed using b
bits, and the quotient is encoded into an unary format. On the other hand, FOR
compress a block of values by applying a compression routine on sub-blocks of
32 integers, where the values across the sub-blocks are encoded into a same
number of bits. By computing the remainders in Rice all together, we achieve the
same use case as FOR, if the number of remainders is a multiple of 32.
The FOR algorithm is only used to compress with CPU optimized routines the
remainders. The compression ratio is the one that we can have thanks to Rice.

The RiceFOR algorithm works as follows. First we compute the remainders given a
parameter $b$ of all values to be compressed. We might have to add some fake
values (i.e., zeros) if the block's size is not a multiple of 32. These
remainders are outputted into a temporary block that will be compressed using
the FOR algorithm with b for the bit frame. Then the quotient of all values
are to be encoded with the unary format, as they would have been with the
original Rice algorithm.

The decompression algorithm works in the same way as the original Rice, the
difference being that the remainders are to be decompressed with FOR.

\subsection{Implementation}

Rice provides a good compression ratio~\cite{zhang:2008:www}, however the
computation of the statistics and unary encoding slow its compression and
decompression speed. Along with the optimization provided by the AFOR highly
performing routines, two optimized algorithms have been developed in order to
encode and decode efficiently values in unary format: an optimized routine to
encode a value into unary format, and a routine to decode unary values at the
byte level inspired from the Rice implementation in \cite{Yan:2009:sigir}.

\begin{description}
\item[Compression]
Using FOR to compress the remainders computed for the Rice
algorithm, RiceFOR is implemented as follows:
\begin{enumerate}
  \item[] Given a block B of integers
  \item the power of 2 directly lower to the mean of B is used as the bit frame.
  \item each sub-blocks of 32 remainders are compressed with that bit frame
  using the FOR routines such as the one presented in the
  Listings~\ref{lst:compression-routine}.
  \item the quotient of each integer (with the division by the bit frame) are
  encoded in unary format, using the optimized routine reported in the
  Listings~\ref{lst:optimized-unary-enc}.
\end{enumerate}
\item[Decompression] The remainders are decompressed with the
FOR routines, and the quotients are decoded using a byte-based decoding method
presented in the Listings~\ref{lst:block-unary-dec}. Indeed a byte is likely to
encode multiple values in unary format (e.g. 4 ones are encoded in one byte as
$0101\;0101$), thus it is possible to decode multiple values by reading a byte
a single time.
\end{description}

In the method \emph{encodeUnary} presented in the
Listings~\ref{lst:optimized-unary-enc}, the loop encode the quotient $q$ into
the array of byte $b$ of compressed data. The variable \emph{bit} is global and
records the number of bit already compressed. On line 3, the local variable
\emph{r} holds the number of bits in the current byte that are still unused.
The variable \emph{ones} holds the number of bits that will be set to 1. The
switch structure displays 8 hard coded cases to write from 1 to 8 bits at 1.
With the example of case 3 on line 7, the current byte, i.e., $bit / 8$, has
3 bits (7 in binary format: $111$) appended to it. These operations will repeat
until the quotient q has been completely encoded. The termination criterion of
the unary encoding, i.e., bit at zero, is computed on line 15. This suggests
that the current byte has been previously initialized to zero.

The method presented in the Listings~\ref{lst:block-unary-dec} reads a byte
containing values in unary format and decode them into an int array, starting
at the position \emph{offset}. All the unary encoded values are quotient
computed by the Rice algorithm, with a same parameter \emph{frameBit}. The
switch structure contains all 256 possibilities of bits configurations, and
hard codes for each cases the corresponding updates. For example the 26 case,
which in binary base equals $0001\;1010$, encodes 5 values which respectively
are from the least to the most significant bit $0$, $1$, $2$, 0 and 0. Thus the
quotients are respectively 0, $1^{bitFrame}$, $2\times1^{bitFrame}$, 0 and 0,
with $bitFrame$ the divisor parameter used in Rice.

\begin{fileformat}
  \centering
  \begin{minipage}[t]{0.47\linewidth}
\begin{lstlisting}[frame=lines,language=Java,numbers=left,caption=Unary
format encoding.,label=lst:optimized-unary-enc]
encodeUnary(int q, byte[] b)
	while (q > 0) {
		int r = 8 - (bit % 8);
		int ones = r > v ? v : r;
		switch (ones) {
			...
			case 3:
				b[bit / 8] |= (7 << (bit % 8));
				bit += 3;
				break;
			...
		}
		q -= ones;
	}
	bit += 1
\end{lstlisting}
  \end{minipage}
  \quad%
  \begin{minipage}[t]{0.47\linewidth}
\begin{lstlisting}[frame=lines,language=Java,numbers=left,caption=Byte-based
unary format decoding.,label=lst:block-unary-dec]
decodeUnary(byte b, int[] i)
	switch (b) {
		...
		case 26: // 26 = 0001 1010
			i[offset] += 0;
			i[offset + 1] += 1^frameBit;
			i[offset + 2] += 2*1^frameBit;
			i[offset + 3] += 0;
			i[offset + 4] += 0;
			offset += 5;
			break;
		...
	}
\end{lstlisting}
  \end{minipage}
\end{fileformat}

\section{Adaptive Frame Of Reference}{
The Adaptive Frame Of Reference (AFOR) attempts to retain the best of FOR,
i.e., a very efficient compression and decompression using highly-optimised
routines to avoid branching conditions, while providing a better tolerance
against outliers and therefore achieving a higher compression ratio. Compared
to PFOR, AFOR does not rely on the encoding of exceptions in the presence of
outliers. Instead, AFOR partitions a block into multiple frames of variable
length, the partition and the length of the frames being chosen appropriately
in order to adapt the encoding to the value distribution.
}
\label{sec:afor}

\subsection{Algorithm}

AFOR extends the FOR algorithm and runs as follows. Given a block $B$ of $n$
integers, AFOR partitions it into $m$ distinct frames and encodes each frame
using highly-optimised routines similarly to FOR. Each frame is independent
from one an other, i.e., each one has its own \emph{bit frame}, and each one
encodes a variable numbers of values. This is depicted in
Figure~\ref{fig:compression:afor} by \emph{AFOR-2}. AFOR encodes along with
the frame the associated bit frame with respect to a given encoder, e.g., a
binary encoder. In fact, AFOR encodes (respectively decodes) a block of values
by:
\begin{enumerate}
\item encoding (respectively decoding) the bit frame;
\item selecting the compression (respectively decompression) routine associated
to the bit frame;
\item encoding (respectively decoding) the frame using the selected routine.
\end{enumerate}

\begin{figure*}
  \centering
	\includegraphics[width=0.7\linewidth]{pics/afor-encoding}
	\caption{Block compression comparison between FOR and AFOR. We alternate
	colours to differentiate independent frames. AFOR-1 denotes a first
	implementation of AFOR using a fixed frame length. AFOR-2 denotes a second
	implementation of AFOR using variable frame lengths. AFOR-3 denotes a third
	implementation using variable frame lengths and the frame stripping technique.
 	\emph{BFS} denotes the byte storing the bit frame selector associated to the
	next frame.}
	\label{fig:compression:afor}
\end{figure*}

Finding the right partitioning, i.e., the optimal configuration of frames and
frame lengths per block, is essential for achieving high compression
ratio~\cite{rossano:2010:vse}. If a frame is too large, the encoding becomes
more sensitive to outliers and wastes bits by using a too big bit frame for
all the other integers. On the contrary, if the frames are too small, the
encoding wastes too much space due to the overhead of storing a larger numbers
of bit frames. The appropriate strategy is to rely 
\begin{enumerate}
  \item on large frames in the presence of a homogeneous sequence of values.
  \item on small frames in the presence of outliers.
\end{enumerate}
Also, alternating
between large and small frames is not only important for achieving high
compression ratio but also for achieving high performance. If the frames are
too small, the encoding has to perform more branching conditions to select the
appropriate routine for each bit frame, and therefore the compression and
decompression performance decrease. Therefore, it is better to rely on large
frames instead of multiple smaller frames when it is possible. Our solution
uses a greedy local optimization algorithm which is explained in the next
section.

\subsection{Partitioning Blocks into Variable Frames}

Finding the optimal configuration of frames and frame lengths for a block of
values is a combinatorial problem. For example, with three different frame
lengths (32, 16 and 8) and a block of size 1024 integers, there are $1.18
\times 10^{30}$ possible combinations. While such a combinatorial problem can
be solved via Dynamic Programming algorithms~\cite{rossano:2010:vse}, the
complexity of such algorithms is still $O(n \times k)$, with the term $n$
being the number of integers and the term $k$ the size of the largest frame,
therefore it greatly impacts the compression performance. Since we are not
only interested by a fast decompression speed and a high compression ratio,
but also by a fast compression speed, this approach is not used in our
experiments. Instead, we use a local optimization algorithm that provides a
satisfactory compression rate while being efficient to efficient to compute.

\begin{description}
\item[Local Optimization algorithm]
AFOR computes the block partitioning by using a sliding window over a block
and determines the optimal configuration of frames and frame lengths for the
current window.

Given a window of size $w$ and a list of possible frame
lengths, we first compute beforehand the possible configurations. For example,
for a window size of 32 and three different frame lengths, 32, 16 and 8, there
are six configurations: $[32],[16,16],[16,8,8],[8,16,8],[8,8,16],[8,8,8,8]$.
Then, given a list of possible configurations for a window, we
estimate the size of each configuration by doing one pass over the values of
the window as shown in the Algorithm~\ref{algo:afor2-opt} (Lines 1-5). This
first step performs $w \times k$ comparisons with $k$ the number of possible
configurations. The second step, Lines 6-12 in Algorithm~\ref{algo:afor2-opt},
performs a pass over the possible configuration and estimates the size of each
configuration in order to find the best one. The \texttt{EstimateSize}
function computes the cost of encoding the window given one configuration,
accounting also the overhead of storing the bit frames. For example, for the
configuration $[8,8,8,8]$ with four frames of size 8 each, and with four
associated bit frames, $b_{1}$ to $b_{4}$, the size of the encoding is
computed as follow: $4 + 8 \times \sum_{i=1\ldots4} log(b_i + 1)$, where $8
\times \sum_{i=1\ldots4} log(b_i + 1)$ is the size of the four encoded frames
and $4$ is the overhead to store the four bit frames.
\end{description}

This simple algorithm is efficient to compute, in particular if the
window size is small and the frame lengths are restricted to a few number.
However, it is easy to see that such method does not provide the optimal
configuration for a complete block. There is a trade-off between optimal
partitioning and complexity of the algorithm. However, one can possibly use a
more complex method for achieving higher compression ratio if the compression
speed is not critical. We decided to use this method since in our experiment
we found that a small window size of 32 values and three frame lengths, 32, 16
and 8, were providing satisfactory results in term of compression speed and
compression ratio. In the next Section is presented a more detailed explanation
of the AFOR implementation.

\SetAlFnt{\sf}
\IncMargin{.5em}
\begin{algorithm}
\SetAlgoLined
\LinesNumbered
\SetKwData{Conf}{c}
\SetKwData{BestConf}{bestConf}\SetKwData{BestSize}{bestSize}
\SetKwInOut{Input}{input}\SetKwInOut{Output}{output}
\SetKwFunction{GetBitFrame}{GetBitFrame}\SetKwFunction{SetBitFrame}{SetBitFrame}
\SetKwArray{BitFrames}{bitFrames}
\SetKwFunction{EstimateSize}{EstimateSize}

\Input{A window $W$ of size $w$}
\Input{The smallest frame length $l$}
\Output{The best configuration for the window}

\BlankLine
\BlankLine

\For{$i \leftarrow 0$ \KwTo $\frac{w}{l}$}{
  \For{$j \leftarrow i \times l$ \KwTo $(i + 1) \times l$}{
    \BitFrames{i} $\leftarrow \max($\BitFrames{i}$, \lceil \log_2(W[j] + 1) \rceil)$\;
  }
}
\BestSize $\leftarrow$ $MaxSize$\;
\ForEach{configuration \Conf of the possible configurations}{
  \If{\EstimateSize{\Conf, \BitFrames} $<$ \BestSize}{
    \BestSize $\leftarrow$ \EstimateSize{\Conf}\;
	\BestConf $\leftarrow$ \Conf\;
  }
}
\caption{The algorithm that finds the best configuration of frames and frame
lengths for a window $W$ of size $w$.}
\label{algo:afor2-opt}
\end{algorithm}
\DecMargin{.5em}

\subsection{Frame Stripping}
\label{sec:compression:stripping}

In an inverted list, it is common to encounter a long sequence of 1 to encode,
i.e., the delta gap value. For example, this occurs with terms that appear
frequently in many entities. With RDF data, such a very common term might be a
predicate URI or a widely present class URI. As a consequence, the list of
entity identifiers is composed of many consecutive identifiers, which is
encoded as a list of 1 using the delta representation. Also, the schema used
across the entity descriptions coming from a same dataset is generally
similar. When indexing batch of entities coming from a same dataset, we
benefit from a ``term clustering'' effect: all the schema terms are associated
with long runs of consecutive entity identifiers in the inverted index. There
is also other cases where a long run of 1 is common, for example in:
\begin{itemize}
\item the list of term frequencies for terms that appear frequently a single
time in the entity description, e.g., class URIs;
\item the list of value identifiers for terms that appear frequently in
single-valued attributes;
\item the list of term positions for nodes holding a single term, e.g., URIs. 
\end{itemize}

In presence of such long runs of 1, AFOR still needs to encode each value
using 1 bit. For example, a frame of 32 values will encode a sequence of 1
using 32 bits. The goal of the \emph{Frame Stripping} method is to avoid the
encoding of such frames. Our solution is to \emph{strip} the content of a
frame if and only if the frame is exclusively composed of 1. We encode such a
case using a special bit frame.

\subsection{Frame Skipping}
\label{sec:afor-skip}

Block-based compression techniques encode the configuration a block of integers
has been compressed with. When the length of the compressed block is unknown,
this compression header possesses information about the block that can be used
to skip this block and so not to decompress it.

With the AFOR algorithm, this method enables the possibility to skip a frame by
reading a BFS. This is not possible with techniques such as Rice, VByte or
PFOR. Since the BFS encodes the bit frame used to compress a frame, the size
in bytes of a compressed frame is found by the formula
\begin{displaymath}
\mid F \mid \times BFS
\end{displaymath}
where $\mid F \mid$ is the length of a frame. For instance the bit frame 1
(i.e., 1 bit) encodes a frame of 32 integers into 4 bytes.

With self-indexing structures (presented in Chapter~\ref{sec:self-indexing}),
the difference in bytes between two points in a stream is necessary in order to
know at which position data has to be read. These \emph{file pointers} have to
be stored, taking a considerable space. Thanks to the frame skipping method, it
is possible to get rid of the additional file pointers information, since the
size in bytes of a frame is known by reading its BFS.

\subsection{Implementations}

We present three different implementations of the AFOR encoder class. We can
obtain many variations of AFOR by using various sets of frame lengths and
different parameters for the partitioning algorithm. We tried many of them
during our experimentation and report here only the ones that are promising
and interesting to compare.

\paragraph{AFOR-1}

The first implementation of AFOR, referred to as AFOR-1 and depicted in
Figure~\ref{fig:compression:afor}, is using a single frame length of 32
values. To clarify, this approach is identical to FOR applied on small blocks
of 32 integers. This first implementation shows the benefits of using short
frames instead of long frames of 1024 values as in our original FOR
implementation. In addition, AFOR-1 is used to compare and judge the benefits
provided by AFOR-2, the second implementation using variable frame lengths.
Considering that, with a fixed frame length, a block is always partitioned in
the same manner, AFOR-1 does not rely on the partitioning algorithm presented
previously.

\paragraph{AFOR-2}

The second implementation, referred to as AFOR-2 and depicted in
Figure~\ref{fig:compression:afor}, relies on three frame lengths: 32, 16 and
8. We found that these three frame lengths give the best balance between
performance and compression ratio. Additional frame lengths were rarely
selected and the performance was decreasing due to the larger number of
partitioning configurations to compute. Reducing the number of possible frame
lengths was providing slightly better performance but slightly worse
compression ratio. There is a trade-off between performance and compression
effectiveness when choosing the right set of frame lengths. Our implementation
relies on the partitioning algorithm presented earlier, using a window's size
of 32 values and six partitioning configurations
$[32],[16,16],[16,8,8],[8,16,8],[8,8,16],[8,8,8,8]$.

\paragraph{AFOR-3}

The third implementation, referred to as AFOR-3 and depicted in
Figure~\ref{fig:compression:afor}, is identical to AFOR-2 but employs the
\emph{frame stripping} technique. Compared to AFOR-2, the compressed block can
contain frames which is then encoded by a single bit frame as depicted in
Figure~\ref{fig:compression:afor}. AFOR-3 implementation relies on the same
partitioning algorithm as AFOR-2 with an additional step to find and strip
frames composed of a sequence of 1 in the partitions.

\subsubsection{Compression and decompression routines}

Our implementations rely on highly-optimised routines such as the ones
presented in Listing~\ref{lst:compression-routine} and
\ref{lst:decompression-routine}, where each routine is loop-unrolled to encode
or decode a fixed number of values using shift and mask operations only. There
is one routine per bit frame and per frame length. For example, with a frame
length of 8 values, the routine encodes 8 values using 3 bits each as shown in
Listing~\ref{lst:compression-routine}, while for a frame length of 32, the
routine encodes 32 values using 3 bits each.

Since AFOR-1 uses a single frame length, it only needs 32 routines for
compression and 32 routines for decompression, i.e., one routine per bit frame
(1 to 32). With respect to AFOR-2, since it relies on three different frame
lengths, it needs 96 routines for compression and 96 routines for
decompression. As for AFOR-3, one additional routine for handling a
sequence of 1 is added per frame length. The associated compression routine is
empty and does nothing since the content of the frame is not encoded.
Therefore the cost is reduced to a single function call. The decompression
routine consists of returning an array of ones. This last routine is very fast
to execute since there are no shift or mask operations.

\subsubsection{Bit frame encoding}

As a reminder the bit frame is encoded along with the frame, so that, at
decompression time, the decoder can read the bit frame and select the
appropriate routine to decode the frame. In the case of AFOR-1, the bit frame
varies between 1 to 32. For AFOR-2, there are 96 cases to be encoded, where
cases 1 to 32 refer to the bit frames for a frame length of 8, cases 33 to 63
for a frame length of 16, and cases 64 to 96 for a frame length of 32. In
AFOR-3, we encode one additional case per frame length with respect to the
frame stripping method. Therefore, there is a total of 99 cases to be encoded.
The cases 97 to 99 refer to a sequence of 1 for a frame length of 8, 16 and 32
respectively.

In our implementation, the bit frame is encoded using one byte. While this
approach wastes some bits each time a bit frame is stored, more precisely 3
bits for AFOR-1 and 1 bit for AFOR-2 and AFOR-3, the choice is again for a
question of efficiency. Since bit frames and frames are interleaved in the
block, storing the bit frame using one full byte enables the frame to be
aligned with the start and the end of a byte boundary. Another implementation
to avoid wasting bits is to pack all the bit frames at the end of the block. We
tried this approach and report that it provides slightly better compression
ratio, but slightly worse performance. Since the interleaved approach was
providing better performance, we decided to use it in our benchmarks.

\subsubsection{Routine selection}

A precomputed lookup table is used by the encoder and decoder to quickly
select the appropriate routine given a bit frame. Compared to AFOR-1, AFOR-2
and AFOR-3 have to perform more table lookups for selecting routines since
they are likely to rely on small frames of 8 or 16 values when the value
distribution is sparse. While these lookups cost additional CPU cycles, we
will see in the experiments that the overhead is minimal.


\chapter{SkipBlock: Self-Indexing for Block-Based Inverted Lists}{%
SkipBlock is a model for block-based inverted lists, that can be applied as
well on traditional inverted lists. It is based on an adaptation of the Skip
List data structure. This work is orthogonal to related work on self-indexing
structures, since they might be extended to this model.

In this section, the SkipBlock model is presented with its associated search
algorithm. We discuss then the impact of the structure's parameters for both
Skip List and SkipBLock. We argue after the different strategies available to
search within intervals for both structures. Finally we compare thanks to a
cost-based model the differences between the Skip Lists and the SkipBlock. }
\label{chap:skipblock}
\section{The SkipBlock Model}
\label{sec:skipbock}

The Skip List model works on a record unit, i.e., a synchronization point has a
probability p to appear in an inverted list of n records.
The SkipBlock model operates instead on \emph{blocks} of records of a fixed
size, in place of the records themselves. Consequently, the probabilistic parameter
$p$ is defined with respect to a block unit. A synchronization point is
created every $\frac{1}{p^i}$ blocks on a level $i$, thus every $\frac{\vert B
\vert}{p^i}$ records where $\vert B \vert$ denotes the block size. A
synchronization point links to the first record of a block interval. Compared
to Figure~\ref{fig:skiplists}, a SkipBlock structure with $p=\frac{1}{8}$ and
$\vert B \vert=2$ also has an interval of $\frac{\vert B \vert}{p^1}=16$
records. However, on level 2, the synchronization points are separated by
$\frac{\vert B \vert}{p^2}=128$ instead of 256 records. We note that with
$\vert B \vert= 1$, the SkipBlock model is equivalent to the original Skip
Lists model, and therefore it is a generalization. The two properties of the
Skip Lists are then translated to this model:
\begin{enumerate}
  \item the number of levels is defined by
\begin{equation}
L_B(n)=\left\lfloor \ln_{\frac{1}{p}}\left(\frac{n}{\vert B \vert}\right) \right\rfloor
\label{eq:skipblock-maxlevels}
\end{equation}
  \item the structure's size
\begin{equation}
S_B(n)=\sum_{i=1}^{
L_B\left(n\right)}{\left\lfloor \frac{n\times p^i}{\vert B \vert}
\right\rfloor}
\label{eq:skipblock-size}
\end{equation}
\end{enumerate}

\subsection{SkipBlock Search Algorithm}
\label{sec:search-skipblock}

With the block-based Skip Lists model, the search algorithm returns an
interval of blocks containing the target record. We discuss the search in
that interval in Section~\ref{sec:search-interval}. The search walk is identical
to the one presented in Section~\ref{sec:search-skiplists}: we walk from the
top to the bottom level, and compare at each step the current synchronization
point with the target. The walk stops at the same termination criteria as in
Skip Lists. The search complexity in the worst case becomes
\begin{equation}
C_{SB}=\frac{L_B(n)}{p}
\label{eq:skipblock-complexity}
\end{equation}

\subsection{Impact of the Self-Indexing Parameters}
\label{sec:block-probas-impact}

In this section, we discuss the consequences of the probabilistic parameter on
the Skip Lists data structure.
Table~\ref{tab:skiplists-proba} reports for low (i.e., $\frac{1}{1024}$) and
high (i.e., $\frac{1}{2}$) probabilities
\begin{inparaenum}[(1)]
\item the complexity (\ref{eq:skiplists-complexity}) to find the interval
containing the target record, and
\item the size (\ref{eq:skiplists-size}) of the Skip Lists structure.
\end{inparaenum}
For example with an interval $\vert I \vert = 1024$ and an
inverted list of size $10^6$ (i.e., Skip Lists with one level), reaching a record
located at the end of the inverted list incurs a lot of synchronization points
at level 1 being read. With a smaller interval, e.g. 16, the number of
synchronization points read decreases because the $L(10^6)=4$ levels provide
more accessing points to the inverted list.

Large intervals give the possibility to skip over a large number of records by
reading a few synchronization points. However this leads to less use cases of
the Skip Lists structure when the searched records are close to each other. In
the latter case, small intervals are then more adapted, but at the cost of a
bigger structure. Moreover the more levels there are, the more walks up and
down on them there are.

There is a trade-off to achieve when selecting $p$: a high probability
provides a low search complexity but at a larger space cost, and a low
probability reduces considerably the required space at the cost of higher
search complexity.

The SkipBlock model provides two parameters to control its Skip Lists
structure: the probabilistic parameter $p$ and the block size $\vert B \vert$.
The block size parameter enables more control over the Skip Lists structure. For
example, to build a structure with an interval of length 64, the original Skip
Lists model proposes only one configuration given by $p=\frac{1}{64}$. For
this same interval length, SkipBlock proposes all the configurations that
verify the equation $\frac{\vert B \vert}{p}=64$.
Table~\ref{tab:skiplists-proba-block} reports statistics of some SkipBlock
configurations for the same interval lengths as in
Table~\ref{tab:skiplists-proba}. Compared to Skip Lists on a same interval
length, SkipBlock shows a lower search complexity in exchange of a larger
structure.

\begin{table}
\begin{center}
\subfloat[Skip Lists with $\vert I\vert=\frac{1}{p}$.]{
\centering
\ra{1.1}
\resizebox{0.39\linewidth}{!}{%
\begin{tabular}{@{}lcc@{\hs}c@{\hs}c@{\hs}c@{\hs}c@{\hs}}
\toprule
$\vert I \vert$ & \phantom{a} & 2 & 16 & 64 & 128 & 1024\\
\cmidrule{2-7}
$S(n)$ & \phantom{a} &
\numprint{99999988}&\numprint{6666664}&\numprint{1587300}&\numprint{787400}&\numprint{97751}\\
$C$ & \phantom{a} & 54 & 112 & 320 & 512 & 3072 \\
\bottomrule
\end{tabular}
\label{tab:skiplists-proba}
}}\quad%
\subfloat[SkipBlock with $\vert I\vert=\frac{\vert B \vert}{p}$.]{
\ra{1.1}
\resizebox{0.54\linewidth}{!}{%
\begin{tabular}{@{}lcc@{\hs}c@{\hs}cc@{\hs}c@{\hs}cc@{\hs}c@{\hs}cc@{\hs}c@{\hs}}
\toprule
$\vert I \vert$ & \phantom{a} & \multicolumn{2}{c}{16} & \phantom{a} &
\multicolumn{2}{c}{64} & \phantom{a} & \multicolumn{2}{c}{128} & \phantom{a} &
\multicolumn{2}{c}{1024} \\
\cmidrule{3-4} \cmidrule{6-7} \cmidrule{9-10} \cmidrule{12-13}
p:$\vert B \vert$ & \phantom{a} & $\frac{1}{4}$:4 & $\frac{1}{8}$:2 & \phantom{a} &
$\frac{1}{4}$:16 & $\frac{1}{8}$:8 & \phantom{a} & $\frac{1}{4}$:32 &
$\frac{1}{8}$:16 & \phantom{a} & $\frac{1}{4}$:256 & $\frac{1}{8}$:128 \\
$S_B(n)$ & \phantom{a} & \numprint{8333328} & \numprint{7142853} &
\phantom{a} & \numprint{2083328} & \numprint{1785710} & \phantom{a} &
\numprint{1041660} & \numprint{892853} & \phantom{a} &
\numprint{130203} & \numprint{111603} \\
$C$ & \phantom{a} & 48 & 64 & \phantom{a} & 44 & 56 & \phantom{a} &
40 & 56 & \phantom{a} & 36 & 48 \\
\bottomrule
\end{tabular}
\label{tab:skiplists-proba-block}
}}%
\end{center}
\caption{Search and size costs of Skip Lists and SkipBlock with $n=10^8$.
$\vert I \vert$ stands for an interval length. $C$ reports the search
complexity to find an interval (Sections~\ref{sec:search-skiplists}
and~\ref{sec:search-skipblock}).}
\end{table}

\section{Searching within an Interval}
\label{sec:search-interval}

The Skip Lists and SkipBlock techniques enables the retrieval of an interval
given a target record. The next step is to find the target record within that
interval. A first strategy (S1) is to linearly scan all the records within
that interval until the target is found. Its complexity is therefore $O(\vert I \vert)$.

SkipBlock takes advantage of the block-based structure of the
interval to perform more efficient search strategy. Here are defined four
additional strategies for searching a block-based interval, parameterized to a
probability p. The second strategy (S2) performs
\begin{enumerate}[(a)]
\item a linear scan over the blocks of the interval to find the block holding
the target and
\label{s2-linear-search-blocks}
\item a linear scan of the records of that block to find the target.
\label{s2-target-block}
\end{enumerate}
The search complexity is $\frac{1}{p}+\vert B \vert$ with $\frac{1}{p}$ denoting
the linear scan over the blocks and $\vert B \vert$ the linear scan over the
records of one block. Similarly to S2, the third strategy (S3) performs the
step (\ref{s2-linear-search-blocks}). Then, it uses to find the target an
inner-block Skip Lists structure restricted to one level only. The complexity
is $\frac{1}{p}+\frac{1}{q}+\left\lfloor\vert B\vert\times q\right\rfloor$
with q the probability of the inner Skip Lists. In contrast to S3, the fourth
strategy (S4) uses a non-restricted inner-block Skip Lists structure. The
complexity is $\frac{1}{p}+\frac{L(\vert B \vert)+1}{q}$ with q the inner Skip
Lists probability. The fifth one (S5) builds a Skip Lists structure on the
whole interval instead of on a block. Its complexity is then
$\frac{L\left(\frac{\vert B\vert}{p}\right)+1}{q}$, with $q$ the inner Skip
Lists probability. The strategies S3, S4 and S5 are equivalent to S2 when the
block size is too small for creating synchronization points.

\section{Cost-Based Comparison}
\label{sec:cost-based-cmp}

In this section, we define a cost model that we use to compare five
SkipBlock implementations and the original Skip Lists implementation.

\paragraph{Cost Model}
\label{sec:skipblock-cost-model}
For both the Skip Lists and the SkipBlock, we define a cost model by
\begin{inparaenum}[(a)]
\item the cost to search for the target, and
\label{cost-search}
\item the cost of the data structure's size.
\label{cost-size}
\end{inparaenum}
The search cost consists in the number of synchronization points traversed to
reach the interval containing the target, plus the number of records scanned
in that interval to find the target. The size cost consists in the total number
of synchronization points in the data structure, including the additional ones
for S3, S4 and S5 in the intervals.

\paragraph{Implementations}
\label{sec:impl} 
We define as the baseline implementation, denoted $I_1$, the Skip
Lists model using the strategy (\emph{S1}). We define
five implementations of the SkipBlock model, denoted by $I_2$, $I_3$,
$I_4$, $I_5$ and $I_6$, based on the five interval search strategies, i.e.,
\emph{S1}, \emph{S2}, \emph{S3},\emph{S4} and \emph{S5} respectively. The
inner Skip Lists in implementations $I_4$, $I_5$ and $I_6$ is
configured with probability $q=\frac{1}{16}$. The inner Skip Lists in $I_5$ and
$I_6$ have at least 2 levels. The size costs of the six implementations are 
\begin{itemize}
  \item[$I_1$:] $S(n)$
  \item[$I_2$:] $S_B(n)$
  \item[$I_3$:] $S_B(n)+\frac{n}{\vert B \vert}$
  \item[$I_4$:] $S_B(n)+\left\lfloor n\times q\right\rfloor$
  \item[$I_5$:] $S_B(n)+\frac{S(\vert B \vert)\times n}{\vert B \vert}$  
  \item[$I_6$:]$S_B(n)+\frac{S(p\times\vert B \vert)\times n}{p\times\vert B
  \vert}$
\end{itemize}

\paragraph{Comparison}

With respect to the SkipBlock model, we tested all the possible
configurations for a given interval length. We report that all of them were
providing better search cost than the baseline. We only report in
Table~\ref{tab:cmp-costs} the configurations providing the best search cost
given an interval length with the associated size cost. We observe that $I_2$
already provides better search cost than the baseline $I_1$ using the same
search strategy \emph{S1}, in exchange of a larger size cost. The other
implementations, i.e., $I_3$, $I_4$, $I_5$ and $I_6$ which use a more
efficient interval search strategies further decrease the search cost. In
addition, their size cost decreases significantly with the size of the
interval. On a large interval ($512$), $I_4$ is able to provide a low search
cost ($64$) while sustaining a small size cost ($6.5e6$). The inner Skip
Lists of $I_5$ and $I_6$ are built on lists too small to provide benefits. To
conclude, $I_4$ seems to provide a good compromise between search cost and
size cost with large intervals.

\begin{table}
\centering
\ra{1.1}
\resizebox{\linewidth}{!}{%
\begin{tabular}{@{}lcc@{\hs}c@{\hs}c@{\hs}c@{\hs}c@{\hs}c@{\hs}cc@{\hs}c@{\hs}c@{\hs}c@{\hs}c@{\hs}c@{\hs}cc@{\hs}c@{\hs}c@{\hs}c@{\hs}c@{\hs}c@{\hs}cc@{\hs}c@{\hs}c@{\hs}c@{\hs}c@{\hs}c@{\hs}}
\toprule
$\vert I\vert$&\phantom{a}&\multicolumn{6}{c}{8}
&\phantom{a}&\multicolumn{6}{c}{16}
&\phantom{a}&\multicolumn{6}{c}{512}
&\phantom{a}&\multicolumn{6}{c}{1152}\\
&\phantom{a}& $I_1$ & $I_2$ & $I_3$ & $I_4$ & $I_5$ & $I_6$  &\phantom{a}& $I_1$ & $I_2$ & $I_3$
& $I_4$ & $I_5$ & $I_6$   &\phantom{a}& $I_1$ & $I_2$ & $I_3$ & $I_4$ & $I_5$ & $I_6$  &\phantom{a}& $I_1$ &
$I_2$ & $I_3$ & $I_4$ & $I_5$ & $I_6$  \\
\cmidrule{3-8}\cmidrule{10-15}\cmidrule{17-22}\cmidrule{24-29}
SC&\phantom{a}& $72$ & $56$ & \multicolumn{4}{c}{$54$}
&\phantom{a}& $112$ & $62$ & \multicolumn{4}{c}{$56$}
&\phantom{a}& $1536$ & $548$ & $120$ & $64$ & $86$ & $84$
&\phantom{a}& $3456$ & $1186$ & $154$ & $74$ & $84$ & $81$ \\
\cmidrule{3-8}\cmidrule{10-15}\cmidrule{17-22}\cmidrule{24-29}
ZC$\times e6$&\phantom{a}& $14.3$ & $16.7$ & \multicolumn{4}{c}{$50.0$}
&\phantom{a}& $6.7$ & $12.5$ & \multicolumn{4}{c}{$25.0$}
&\phantom{a}& $0.2$ & $0.3$ &$1.8$ & $6.5$ & $7.0$ & $6.9$
&\phantom{a}& $0.09$ &$0.17$ & $1.7$&$6.4$ & $6.6$&$6.7$\\
\bottomrule
\end{tabular}
}%
\caption{Search (i.e., SC) and size (i.e., ZC, in million) costs with $n=10^8$.
SkipBlock implementations report the best search cost with the associated
size cost.}
\label{tab:cmp-costs}
\end{table}

\subsection{Skipping Analysis}

The SkipBlock structure, as a generalization of the Skip List, possesses the
same hierarchical layout. However the difference between the two is concrete:
SkipBlock jumps over groups of blocks at a time, whereas Skip Lists jumps over
groups of records. The Table~\ref{tab:skip-levels} reports the intervals at
each levels for both structures, with a constant interval between towers of
256 records. For a same interval length (e.g., 256 records in the table), we are
able to yield more configurations with the SkipBlock than with the Skip List
model. We remark that in both configurations of the SkipBlock, in-between
levels have been added in comparison to the original Skip List. These
additional levels give more skipping possibilities while searching. Moreover
this implies that more data will be read from a tower in a SkipBlock structure
than with the Skip List, for a same interval length.

\begin{table}
\ra{1.1}
\centering
\resizebox{0.45\linewidth}{!}{%
\begin{tabular}{lc@{\hs}rc@{\hs}rr}
\toprule
Level & \phantom{a} & Skip Lists & \phantom{a} & \multicolumn{2}{c}{SkipBlock}\\
\cmidrule{5-6}
& & p=256 & \phantom{a} & B=16 p=16 & B=64 p=4 \\
\cmidrule{5-6}
1 & \phantom{a} & 256 & \phantom{a} & 256 & 256 \\
2 & \phantom{a} & \uwave{\numprint{65536}} & \phantom{a} & \numprint{4096} &
\numprint{1024} \\
3 & \phantom{a} & \uwave{\numprint{16777216}} & \phantom{a} &
\uwave{\numprint{65536}} & \numprint{4096} \\
4 & \phantom{a} & \ldots & \phantom{a} & \numprint{1048576} & \numprint{16384}\\
5 & \phantom{a} & \ldots & \phantom{a} & \uwave{\numprint{16777216}} &
\uwave{\numprint{65536}}\\
\bottomrule
\end{tabular}
}
\caption{Interval length for each skip levels.}
\label{tab:skip-levels}
\end{table}

The SkipBlock structure will behave differently depending on the parameter
configurations in comparison to the Skip list, since they yield different skip
levels. To have a better understanding of the differences between the two
models, we discuss here the cost of reading skip data into memory and in which
case it outweighs the CPU saving thanks to an adaptation of the method from
\cite{moffat:96}.

Let $t_r$ be the cost of reading one record or a synchronization point, both as
part of a bulk read. Let also $t_d$ be the cost of decoding a
record/synchronization point. Let further $k$ be the number of
\emph{candidates}, i.e., the records that will be lookup from the inverted
list thanks to the self-indexing structure. Then the total time $T$ required
to search one inverted list of size $n$ in the worst case scenario (i.e., all
levels up to the maximum $L(n)$ are read) is given by the formula
\begin{displaymath}
T=T_d+T_r=kt_d\left( C + \vert I \vert \right)+t_r\left(n+2\times S\right) 
\end{displaymath}
where $T_d$ is the total time to decode pointers, $T_r$ the total time to read
into memory the inverted list and synchronization points. $ZC$ and $SC$ reports
respectively the size and the search costs. The numerical values of $t_d$ and
$t_r$ are respectively $2.5\times 10^{-6}$ and $0.5\times 10^{-6}$, values
taken from \cite{moffat:96}.

The Table~\ref{tab:predicted-times} reports a numerical comparison between the
two self-indexing models, with $n=60,000$ and $k=\{60,\;6000\}$. With a bigger
number of candidates we can expect a bigger processing time, since the
self-indexing structure will be accessed more often. The SkipBLock structure's
values are reported with the strategy S1 for the configurations that yields
the best processing time T. On small intervals (e.g., 16) the SkipBlock model
does not provide significant better performance even though the complexity is
better than the Skip List. This is explained by the difference in size, as the
amount of data to be read is greater, which is reflected by the greater
required time to read data $T_r$. As the interval increases, we can see that
the processing times increase for both models. However the expected
processing time is inferior with SkipBlock than with the original Skip List by
2 times.
The reason is that even if the strategies for searching within intervals is
the same (i.e., linear), the additional levels that the SkipBlock model
provides give more skipping possibilities. This comforts the fact that the
SkipBlock complexities are lower than the Skip List's. With the interval
$\vert I \vert = 1024$, the structure's size increases by 30\%, however both
the search complexity and the estimated processing time are twice as low as
the original model.

\begin{table}
\ra{1.1}
\centering
\resizebox{\linewidth}{!}{%
\begin{tabular}{lc@{\hs}rrrc@{\hs}rrrc@{\hs}rrc@{\whs}rrrc@{\hs}rrrc@{\hs}rr}
\toprule
$\vert I \vert$ & \phantom{a} & \multicolumn{10}{c}{Skip List} & \phantom{a} &
\multicolumn{10}{c}{SkipBlock} \\
\cmidrule{3-12} \cmidrule{14-23}
k &\phantom{a}& \multicolumn{3}{c}{60} &\phantom{a}& \multicolumn{3}{c}{6000}
&\phantom{a}& & & \phantom{a} & \multicolumn{3}{c}{60} &\phantom{a}&
\multicolumn{3}{c}{6000} & & \\
\cmidrule{3-5} \cmidrule{7-9} \cmidrule{14-16} \cmidrule{18-20}
& \phantom{a} & $T_d$ & $T_r$ & T & \phantom{a} & $T_d$ & $T_r$ & T &
\phantom{a} & SC & ZC & \phantom{a} & $T_d$ & $T_r$ & T & \phantom{a} &
$T_d$ & $T_r$ & T & \phantom{a} & SC & ZC \\
16&\phantom{a}&0.010&0.034&0.044
&\phantom{a}& 0.960 & 0.034 & 0.994 & \phantom{a}& 64 & 3998
&\phantom{a}&0.006&0.035&0.041
&\phantom{a}& 0.600 & 0.035 & 0.635 & \phantom{a}&40 & 4996 \\

32&\phantom{a}&0.019&0.032&0.051
&\phantom{a}& 1.92 & 0.032 & 1.952& \phantom{a} & 128 & 1934
&\phantom{a}&0.008&0.034&0.042
&\phantom{a}& 0.81 & 0.034 & 0.844 & \phantom{a}& 54 & 3743 \\

256&\phantom{a}&0.077&0.030&0.107
&\phantom{a}& 7.68 & 0.030 & 7.710& \phantom{a} & 512 & 234
&\phantom{a}&0.041&0.030&0.071
&\phantom{a}& 4.08 & 0.030 & 4.110& \phantom{a} & 272& 309 \\

512&\phantom{a}&0.154&0.030&0.184
&\phantom{a}& 15.36 & 0.030 & 15.390& \phantom{a} & 1024 & 117 
&\phantom{a}&0.079&0.030&0.109
&\phantom{a}& 7.89 & 0.030 & 7.920 & \phantom{a}& 526 & 229 \\

1024&\phantom{a}&0.307&0.030&0.337
&\phantom{a}&30.72 & 0.030 & 30.750& \phantom{a} & 2048 & 58
&\phantom{a}&0.155&0.030&0.185
&\phantom{a}& 15.54 & 0.030 & 15.570 & \phantom{a}& 1036 & 75 \\

\bottomrule
\end{tabular}}
\caption{Predicted processing times in seconds (n=60,000), with the
corresponding structure's search (i.e., SC) and size (i.e., ZC) costs. $T_d$
stands for the time for decoding a record, $T_r$ for the time of reading into memory a
record/synchronization point and T the sum of these two times. $k$ indicate the
number of candidates.}
\label{tab:predicted-times}
\end{table}
