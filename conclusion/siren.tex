\section{NTriple Query Parser}
\label{sec:N3-qparser}

As with the SPARQL language, the NTriple query parser aims to find sub-graph
within datasets that best match a given set of triple patterns. Triple patterns
are linked together thanks to Boolean operators, i.e. AND, OR, NOT. To express
such patterns, SIREn uses a bottom-up grammar, and is built as a plugin to the
Lucene\footnote{Lucene: \url{http://lucene.apache.org/}} query parser.
\begin{description}
  \item[Literals] any string enclosed between simple quotes are taken as a
  literal.
  \item[Literal Pattern] any string enclosed between double quotes. It
  describes the pattern that has to appear within the object of the returned
  triples for the query, thanks to Boolean operations.
  \item[URIs] any string enclosed between the characters '$<$' \whs '$>$'.
\end{description}
Any part of a triple pattern, to a maximum of two per triple, can be left
unknown thanks to the wildcard character '*'. Any term of the query can be
flagged in order to indicate if the term has to, must not or may appear within
the triples in the results by using respectively the characters '+', '-', ''
(i.e no character). Considering URIs are tokenism when indexing (e.g.
\url{http://sindice.com/} is tokenism into ``HTTP'', ``Sindice'' and ``com''),
the following query
\begin{quotation}
$<$rings$>$ $<$author$>$ ``j.r.r.\_tolkien OR tolkien'' .
\end{quotation}
matches the RDF graph of the Figure~\ref{fig:rdf-graph}. The object of this
triple pattern is a literal pattern, which aims to widen the search to several
writings of the author's name. Since SIREn is an Information Retrieval search
engine, it processes the query as explained in Chapter~\ref{sec:IR} with set
theory operations (i.e. conjunction, disjunction and exclusion).

During indexing, it is possible to store into a specific field any text. For
instance, one can have three different fields such as ``title'', ``author'' and
``abstract'' for the entities in the Figure~\ref{fig:entities}. This way the
abstract of the entity \emph{\_:bnode1} would fall into its field, allowing for
more precise search. A field is identified with the special Lucene character
semi-colon, ':'. For instance the term ``hobbit'' within the entity \_:bnode1
can be refer ed as \emph{abstract{\bfseries :}hobbit}. 

\subsection{Added Features}

The SIREn query parser has been extended to support commonly used feature in
RDF and also to be more expressive. In this section three added features are
presented.

\paragraph{URI Pattern}

As with literal patterns, this extension allows to use Boolean operators within
an URI. It is then possible for example to search for RDF graphs where some
URI(s) contains both the terms \emph{lord} and \emph{rings} by entering the
query \emph{$<$lord AND rings$>$ * * .}$\;$.

\paragraph{Qualified Name - QName}

As explained in the introduction to Information Retrieval in
Chapter~\ref{sec:IR}, a QName is a term that is used to abbreviate an URI.
Whiting an URI, it is possible to use a QName by appending the term to the
semi-colon character. The QName is replaced at query processing time by the
correct namespace thanks to a file containing the mapping between a QName and
its expansion.
This expansion is performed on any string that matches a QName format only
within an URI, and in that case leaves the term unchanged if no mapping key has
been found. Restricting the substitution to URIs allows to QNames-like strings
within literals.

A QName can possess several expansion, e.g. the web
site\footnote{QNames namespace lookup for developers:
\url{http://prefix.cc/}} reports that the QName \emph{dbpedia} is mapped to
three possible expansion
\begin{enumerate}
  \item \url{http://dbpedia.org/resource/}
  \item \url{http://dbpedia.org/dbprop/}
  \item \url{http://dbpedia.org/property/}
\end{enumerate}
In order to deal with multi-valued QNames, all possible expansion replace their
QName, linked together with the Boolean operation OR. For instance the query
\begin{quotation}
* $<$\textcolor{red}{dbpedia}:\textcolor{blue}{title}$>$ * .
\end{quotation}
is expanded to
\begin{quotation}
\resizebox{\linewidth}{!}{%
* $<$\textcolor{red}{\url{http://dbpedia.org/resource/}}
\textcolor{blue}{title} OR
\textcolor{red}{\url{http://dbpedia.org/dbprop/}}\textcolor{blue}{title} OR
\textcolor{red}{\url{http://dbpedia.org/property/}}\textcolor{blue}{title}$>$ * .
}%
\end{quotation}
This expansion is rendered possible thanks to the URI pattern extension of an
URI presented previously. Thus it allows to perform more generic search and to
abstract from the actual schema definition the RDF graphs possess.

\paragraph{Multi-field search}

SIREn is based on Lucene, which data representation model uses fields. Fields
allows to organize logically the information. For example we can split the
documents from the collection according to their dataset, so that all documents
from a same dataset are within the same field.

However spitting information into fields implies that we lose a global view of
the data. To cope with this drawback, we have implemented an extension to the
query parser that search documents across fields, enabling the retrieval of
triple patterns regardless of the field it actually occurs. Moreover depending
on the reliability or the importance that is given to field, it is possible to
affect a \emph{weight} value to that field.

As an example, below is shown a RDF graph with two triples taken from two
different datasets: ``dataset1'' and ``dataset2''
\begin{quotation}
dataset1: $<$http://s$>$ $<$http://p1$>$ ``literal'' .

dataset2: $<$http://s$>$ $<$http://p2$>$ $<$http://o2$>$ .
\end{quotation}
While the query with two triple patterns
\begin{quotation}
($<$http://s$>$ * 'literal' .) AND ($<$http://s$>$ *
$<$\textcolor{blue}{http://o2}$>$ .)
\end{quotation}
matches the previous graph thanks to the implicit link, i.e., the first pattern
returns the triple from dataset1 and the second one from dataset2, the query
\begin{quotation}
($<$http://s$>$ * 'literal' .) AND ($<$http://s$>$ *
$<$\textcolor{blue}{http://o1}$>$ .)
\end{quotation}
does not, since the second pattern cannot be found.

\section{Entity Query Parser}
\label{sec:ent-qparser}

With the semantic web representation model such as RDF, we can describe people,
products, countries or any other entities. When searching for information we
often want results that are relevant for a specific entity, any other results
polluting the information retrieved. As a first step, SIREn is indexing
entities with an entity-centric rather than a document-centric model. The
second step is to build a parser that handles queries for an entity.

The entity query parser is being developed which goal is to matcher patterns for
an entity. For example we will able to formulate the query ``Find all the
people within DERI that are from France''. To better represent relations
between an entity and other entities or objects, we use a one-to-many
representation model called the \emph{tabular} format. The
Table~\ref{tab:tabular-format} depicts triples in a document-centric
representation on the left, and its tabular view on the right. With the tabular
format we define as a field the predicate (i.e., attribute) and store within
that field all the related objects. To answer the previous example query, we
can have two fields in the entity description of DERI, the countries and the
people fields.

For such queries to be efficient the schema of the data has to be known. As
such this parser is intended to be use over specific datasets, such as the
intranet of a company.

\begin{table}
\ra{0.3}
\centering
\begin{tabular}{lllc@{\hs}ll}
\toprule
\multicolumn{3}{c}{Document-centric} & \phantom{a} & \multicolumn{2}{c}{Tabular
format} \\
$<$ s $>$ & $<$ p1 $>$ & $<$ o1 $>\;.$ & \phantom{a} &
\multirow{2}{*}{$<$ o1 $> \; <$ o2 \; o3 $>$} \\
$<$ s $>$ & $<$ p2 $>$ & $<$ o2 $> \; <$ o3 $>\;.$ & \phantom{a} & & \\
\bottomrule
\end{tabular}
\caption{The tabular format data representation.}
\label{tab:tabular-format}
\end{table}
