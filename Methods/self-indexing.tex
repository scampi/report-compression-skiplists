\section{The SkipBlock Model}
\label{sec:skipbock}

The Skip List model works on a record unit, i.e., a synchronization point has a
probability p to appear in an inverted list of n records.
The SkipBlock model operates instead on \emph{blocks} of records of a fixed
size, in place of the records themselves. Consequently, the probabilistic parameter
$p$ is defined with respect to a block unit. A synchronization point is
created every $\frac{1}{p^i}$ blocks on a level $i$, thus every $\frac{\vert B
\vert}{p^i}$ records where $\vert B \vert$ denotes the block size. A
synchronization point links to the first record of a block interval. Compared
to Figure~\ref{fig:skiplists}, a SkipBlock structure with $p=\frac{1}{8}$ and
$\vert B \vert=2$ also has an interval of $\frac{\vert B \vert}{p^1}=16$
records. However, on level 2, the synchronization points are separated by
$\frac{\vert B \vert}{p^2}=128$ instead of 256 records. We note that with
$\vert B \vert= 1$, the SkipBlock model is equivalent to the original Skip
Lists model, and therefore it is a generalization. The two properties of the
Skip Lists are then translated to this model:
\begin{enumerate}
  \item the number of levels is defined by
\begin{equation}
L_B(n)=\left\lfloor \ln_{\frac{1}{p}}\left(\frac{n}{\vert B \vert}\right) \right\rfloor
\label{eq:skipblock-maxlevels}
\end{equation}
  \item the structure's size
\begin{equation}
S_B(n)=\sum_{i=1}^{
L_B\left(n\right)}{\left\lfloor \frac{n\times p^i}{\vert B \vert}
\right\rfloor}
\label{eq:skipblock-size}
\end{equation}
\end{enumerate}

\subsection{SkipBlock Search Algorithm}
\label{sec:search-skipblock}

With the block-based Skip Lists model, the search algorithm returns an
interval of blocks containing the target record. We discuss the search in
that interval in Section~\ref{sec:search-interval}. The search walk is identical
to the one presented in Section~\ref{sec:search-skiplists}: we walk from the
top to the bottom level, and compare at each step the current synchronization
point with the target. The walk stops at the same termination criteria as in
Skip Lists. The search complexity in the worst case becomes
\begin{equation}
C_{SB}=\frac{L_B(n)}{p}
\label{eq:skipblock-complexity}
\end{equation}

\subsection{Impact of the Self-Indexing Parameters}
\label{sec:block-probas-impact}

In this section, we discuss the consequences of the probabilistic parameter on
the Skip Lists data structure.
Table~\ref{tab:skiplists-proba} reports for low (i.e., $\frac{1}{1024}$) and
high (i.e., $\frac{1}{2}$) probabilities
\begin{inparaenum}[(1)]
\item the complexity (\ref{eq:skiplists-complexity}) to find the interval
containing the target record, and
\item the size (\ref{eq:skiplists-size}) of the Skip Lists structure.
\end{inparaenum}
For example with an interval $\vert I \vert = 1024$ and an
inverted list of size $10^6$ (i.e., Skip Lists with one level), reaching a record
located at the end of the inverted list incurs a lot of synchronization points
at level 1 being read. With a smaller interval, e.g. 16, the number of
synchronization points read decreases because the $L(10^6)=4$ levels provide
more accessing points to the inverted list.

Large intervals give the possibility to skip over a large number of records by
reading a few synchronization points. However this leads to less use cases of
the Skip Lists structure when the searched records are close to each other. In
the latter case, small intervals are then more adapted, but at the cost of a
bigger structure. Moreover the more levels there are, the more walks up and
down on them there are.

There is a trade-off to achieve when selecting $p$: a high probability
provides a low search complexity but at a larger space cost, and a low
probability reduces considerably the required space at the cost of higher
search complexity.

The SkipBlock model provides two parameters to control its Skip Lists
structure: the probabilistic parameter $p$ and the block size $\vert B \vert$.
The block size parameter enables more control over the Skip Lists structure. For
example, to build a structure with an interval of length 64, the original Skip
Lists model proposes only one configuration given by $p=\frac{1}{64}$. For
this same interval length, SkipBlock proposes all the configurations that
verify the equation $\frac{\vert B \vert}{p}=64$.
Table~\ref{tab:skiplists-proba-block} reports statistics of some SkipBlock
configurations for the same interval lengths as in
Table~\ref{tab:skiplists-proba}. Compared to Skip Lists on a same interval
length, SkipBlock shows a lower search complexity in exchange of a larger
structure.

\begin{table}
\begin{center}
\subfloat[Skip Lists with $\vert I\vert=\frac{1}{p}$.]{
\centering
\ra{1.1}
\resizebox{0.39\linewidth}{!}{%
\begin{tabular}{@{}lcc@{\hs}c@{\hs}c@{\hs}c@{\hs}c@{\hs}}
\toprule
$\vert I \vert$ & \phantom{a} & 2 & 16 & 64 & 128 & 1024\\
\cmidrule{2-7}
$S(n)$ & \phantom{a} &
\numprint{99999988}&\numprint{6666664}&\numprint{1587300}&\numprint{787400}&\numprint{97751}\\
$C$ & \phantom{a} & 54 & 112 & 320 & 512 & 3072 \\
\bottomrule
\end{tabular}
\label{tab:skiplists-proba}
}}\quad%
\subfloat[SkipBlock with $\vert I\vert=\frac{\vert B \vert}{p}$.]{
\ra{1.1}
\resizebox{0.54\linewidth}{!}{%
\begin{tabular}{@{}lcc@{\hs}c@{\hs}cc@{\hs}c@{\hs}cc@{\hs}c@{\hs}cc@{\hs}c@{\hs}}
\toprule
$\vert I \vert$ & \phantom{a} & \multicolumn{2}{c}{16} & \phantom{a} &
\multicolumn{2}{c}{64} & \phantom{a} & \multicolumn{2}{c}{128} & \phantom{a} &
\multicolumn{2}{c}{1024} \\
\cmidrule{3-4} \cmidrule{6-7} \cmidrule{9-10} \cmidrule{12-13}
p:$\vert B \vert$ & \phantom{a} & $\frac{1}{4}$:4 & $\frac{1}{8}$:2 & \phantom{a} &
$\frac{1}{4}$:16 & $\frac{1}{8}$:8 & \phantom{a} & $\frac{1}{4}$:32 &
$\frac{1}{8}$:16 & \phantom{a} & $\frac{1}{4}$:256 & $\frac{1}{8}$:128 \\
$S_B(n)$ & \phantom{a} & \numprint{8333328} & \numprint{7142853} &
\phantom{a} & \numprint{2083328} & \numprint{1785710} & \phantom{a} &
\numprint{1041660} & \numprint{892853} & \phantom{a} &
\numprint{130203} & \numprint{111603} \\
$C$ & \phantom{a} & 48 & 64 & \phantom{a} & 44 & 56 & \phantom{a} &
40 & 56 & \phantom{a} & 36 & 48 \\
\bottomrule
\end{tabular}
\label{tab:skiplists-proba-block}
}}%
\end{center}
\caption{Search and size costs of Skip Lists and SkipBlock with $n=10^8$.
$\vert I \vert$ stands for an interval length. $C$ reports the search
complexity to find an interval (Sections~\ref{sec:search-skiplists}
and~\ref{sec:search-skipblock}).}
\end{table}

\section{Searching within an Interval}
\label{sec:search-interval}

The Skip Lists and SkipBlock techniques enables the retrieval of an interval
given a target record. The next step is to find the target record within that
interval. A first strategy (S1) is to linearly scan all the records within
that interval until the target is found. Its complexity is therefore $O(\vert I \vert)$.

SkipBlock takes advantage of the block-based structure of the
interval to perform more efficient search strategy. Here are defined four
additional strategies for searching a block-based interval, parameterized to a
probability p. The second strategy (S2) performs
\begin{enumerate}[(a)]
\item a linear scan over the blocks of the interval to find the block holding
the target and
\label{s2-linear-search-blocks}
\item a linear scan of the records of that block to find the target.
\label{s2-target-block}
\end{enumerate}
The search complexity is $\frac{1}{p}+\vert B \vert$ with $\frac{1}{p}$ denoting
the linear scan over the blocks and $\vert B \vert$ the linear scan over the
records of one block. Similarly to S2, the third strategy (S3) performs the
step (\ref{s2-linear-search-blocks}). Then, it uses to find the target an
inner-block Skip Lists structure restricted to one level only. The complexity
is $\frac{1}{p}+\frac{1}{q}+\left\lfloor\vert B\vert\times q\right\rfloor$
with q the probability of the inner Skip Lists. In contrast to S3, the fourth
strategy (S4) uses a non-restricted inner-block Skip Lists structure. The
complexity is $\frac{1}{p}+\frac{L(\vert B \vert)+1}{q}$ with q the inner Skip
Lists probability. The fifth one (S5) builds a Skip Lists structure on the
whole interval instead of on a block. Its complexity is then
$\frac{L\left(\frac{\vert B\vert}{p}\right)+1}{q}$, with $q$ the inner Skip
Lists probability. The strategies S3, S4 and S5 are equivalent to S2 when the
block size is too small for creating synchronization points.

\section{Cost-Based Comparison}
\label{sec:cost-based-cmp}

In this section, we define a cost model that we use to compare five
SkipBlock implementations and the original Skip Lists implementation.

\paragraph{Cost Model}
\label{sec:skipblock-cost-model}
For both the Skip Lists and the SkipBlock, we define a cost model by
\begin{inparaenum}[(a)]
\item the cost to search for the target, and
\label{cost-search}
\item the cost of the data structure's size.
\label{cost-size}
\end{inparaenum}
The search cost consists in the number of synchronization points traversed to
reach the interval containing the target, plus the number of records scanned
in that interval to find the target. The size cost consists in the total number
of synchronization points in the data structure, including the additional ones
for S3, S4 and S5 in the intervals.

\paragraph{Implementations}
\label{sec:impl} 
We define as the baseline implementation, denoted $I_1$, the Skip
Lists model using the strategy (\emph{S1}). We define
five implementations of the SkipBlock model, denoted by $I_2$, $I_3$,
$I_4$, $I_5$ and $I_6$, based on the five interval search strategies, i.e.,
\emph{S1}, \emph{S2}, \emph{S3},\emph{S4} and \emph{S5} respectively. The
inner Skip Lists in implementations $I_4$, $I_5$ and $I_6$ is
configured with probability $q=\frac{1}{16}$. The inner Skip Lists in $I_5$ and
$I_6$ have at least 2 levels. The size costs of the six implementations are 
\begin{itemize}
  \item[$I_1$:] $S(n)$
  \item[$I_2$:] $S_B(n)$
  \item[$I_3$:] $S_B(n)+\frac{n}{\vert B \vert}$
  \item[$I_4$:] $S_B(n)+\left\lfloor n\times q\right\rfloor$
  \item[$I_5$:] $S_B(n)+\frac{S(\vert B \vert)\times n}{\vert B \vert}$  
  \item[$I_6$:]$S_B(n)+\frac{S(p\times\vert B \vert)\times n}{p\times\vert B
  \vert}$
\end{itemize}

\paragraph{Comparison}

With respect to the SkipBlock model, we tested all the possible
configurations for a given interval length. We report that all of them were
providing better search cost than the baseline. We only report in
Table~\ref{tab:cmp-costs} the configurations providing the best search cost
given an interval length with the associated size cost. We observe that $I_2$
already provides better search cost than the baseline $I_1$ using the same
search strategy \emph{S1}, in exchange of a larger size cost. The other
implementations, i.e., $I_3$, $I_4$, $I_5$ and $I_6$ which use a more
efficient interval search strategies further decrease the search cost. In
addition, their size cost decreases significantly with the size of the
interval. On a large interval ($512$), $I_4$ is able to provide a low search
cost ($64$) while sustaining a small size cost ($6.5e6$). The inner Skip
Lists of $I_5$ and $I_6$ are built on lists too small to provide benefits. To
conclude, $I_4$ seems to provide a good compromise between search cost and
size cost with large intervals.

\begin{table}
\centering
\ra{1.1}
\resizebox{\linewidth}{!}{%
\begin{tabular}{@{}lcc@{\hs}c@{\hs}c@{\hs}c@{\hs}c@{\hs}c@{\hs}cc@{\hs}c@{\hs}c@{\hs}c@{\hs}c@{\hs}c@{\hs}cc@{\hs}c@{\hs}c@{\hs}c@{\hs}c@{\hs}c@{\hs}cc@{\hs}c@{\hs}c@{\hs}c@{\hs}c@{\hs}c@{\hs}}
\toprule
$\vert I\vert$&\phantom{a}&\multicolumn{6}{c}{8}
&\phantom{a}&\multicolumn{6}{c}{16}
&\phantom{a}&\multicolumn{6}{c}{512}
&\phantom{a}&\multicolumn{6}{c}{1152}\\
&\phantom{a}& $I_1$ & $I_2$ & $I_3$ & $I_4$ & $I_5$ & $I_6$  &\phantom{a}& $I_1$ & $I_2$ & $I_3$
& $I_4$ & $I_5$ & $I_6$   &\phantom{a}& $I_1$ & $I_2$ & $I_3$ & $I_4$ & $I_5$ & $I_6$  &\phantom{a}& $I_1$ &
$I_2$ & $I_3$ & $I_4$ & $I_5$ & $I_6$  \\
\cmidrule{3-8}\cmidrule{10-15}\cmidrule{17-22}\cmidrule{24-29}
SC&\phantom{a}& $72$ & $56$ & \multicolumn{4}{c}{$54$}
&\phantom{a}& $112$ & $62$ & \multicolumn{4}{c}{$56$}
&\phantom{a}& $1536$ & $548$ & $120$ & $64$ & $86$ & $84$
&\phantom{a}& $3456$ & $1186$ & $154$ & $74$ & $84$ & $81$ \\
\cmidrule{3-8}\cmidrule{10-15}\cmidrule{17-22}\cmidrule{24-29}
ZC$\times e6$&\phantom{a}& $14.3$ & $16.7$ & \multicolumn{4}{c}{$50.0$}
&\phantom{a}& $6.7$ & $12.5$ & \multicolumn{4}{c}{$25.0$}
&\phantom{a}& $0.2$ & $0.3$ &$1.8$ & $6.5$ & $7.0$ & $6.9$
&\phantom{a}& $0.09$ &$0.17$ & $1.7$&$6.4$ & $6.6$&$6.7$\\
\bottomrule
\end{tabular}
}%
\caption{Search (i.e., SC) and size (i.e., ZC, in million) costs with $n=10^8$.
SkipBlock implementations report the best search cost with the associated
size cost.}
\label{tab:cmp-costs}
\end{table}

\subsection{Skipping Analysis}

The SkipBlock structure, as a generalization of the Skip List, possesses the
same hierarchical layout. However the difference between the two is concrete:
SkipBlock jumps over groups of blocks at a time, whereas Skip Lists jumps over
groups of records. The Table~\ref{tab:skip-levels} reports the intervals at
each levels for both structures, with a constant interval between towers of
256 records. For a same interval length (e.g., 256 records in the table), we are
able to yield more configurations with the SkipBlock than with the Skip List
model. We remark that in both configurations of the SkipBlock, in-between
levels have been added in comparison to the original Skip List. These
additional levels give more skipping possibilities while searching. Moreover
this implies that more data will be read from a tower in a SkipBlock structure
than with the Skip List, for a same interval length.

\begin{table}
\ra{1.1}
\centering
\resizebox{0.45\linewidth}{!}{%
\begin{tabular}{lc@{\hs}rc@{\hs}rr}
\toprule
Level & \phantom{a} & Skip Lists & \phantom{a} & \multicolumn{2}{c}{SkipBlock}\\
\cmidrule{5-6}
& & p=256 & \phantom{a} & B=16 p=16 & B=64 p=4 \\
\cmidrule{5-6}
1 & \phantom{a} & 256 & \phantom{a} & 256 & 256 \\
2 & \phantom{a} & \uwave{\numprint{65536}} & \phantom{a} & \numprint{4096} &
\numprint{1024} \\
3 & \phantom{a} & \uwave{\numprint{16777216}} & \phantom{a} &
\uwave{\numprint{65536}} & \numprint{4096} \\
4 & \phantom{a} & \ldots & \phantom{a} & \numprint{1048576} & \numprint{16384}\\
5 & \phantom{a} & \ldots & \phantom{a} & \uwave{\numprint{16777216}} &
\uwave{\numprint{65536}}\\
\bottomrule
\end{tabular}
}
\caption{Interval length for each skip levels.}
\label{tab:skip-levels}
\end{table}

The SkipBlock structure will behave differently depending on the parameter
configurations in comparison to the Skip list, since they yield different skip
levels. To have a better understanding of the differences between the two
models, we discuss here the cost of reading skip data into memory and in which
case it outweighs the CPU saving thanks to an adaptation of the method from
\cite{moffat:96}.

Let $t_r$ be the cost of reading one record or a synchronization point, both as
part of a bulk read. Let also $t_d$ be the cost of decoding a
record/synchronization point. Let further $k$ be the number of
\emph{candidates}, i.e., the records that will be lookup from the inverted
list thanks to the self-indexing structure. Then the total time $T$ required
to search one inverted list of size $n$ in the worst case scenario (i.e., all
levels up to the maximum $L(n)$ are read) is given by the formula
\begin{displaymath}
T=T_d+T_r=kt_d\left( C + \vert I \vert \right)+t_r\left(n+2\times S\right) 
\end{displaymath}
where $T_d$ is the total time to decode pointers, $T_r$ the total time to read
into memory the inverted list and synchronization points. $ZC$ and $SC$ reports
respectively the size and the search costs. The numerical values of $t_d$ and
$t_r$ are respectively $2.5\times 10^{-6}$ and $0.5\times 10^{-6}$, values
taken from \cite{moffat:96}.

The Table~\ref{tab:predicted-times} reports a numerical comparison between the
two self-indexing models, with $n=60,000$ and $k=\{60,\;6000\}$. With a bigger
number of candidates we can expect a bigger processing time, since the
self-indexing structure will be accessed more often. The SkipBLock structure's
values are reported with the strategy S1 for the configurations that yields
the best processing time T. On small intervals (e.g., 16) the SkipBlock model
does not provide significant better performance even though the complexity is
better than the Skip List. This is explained by the difference in size, as the
amount of data to be read is greater, which is reflected by the greater
required time to read data $T_r$. As the interval increases, we can see that
the processing times increase for both models. However the expected
processing time is inferior with SkipBlock than with the original Skip List by
2 times.
The reason is that even if the strategies for searching within intervals is
the same (i.e., linear), the additional levels that the SkipBlock model
provides give more skipping possibilities. This comforts the fact that the
SkipBlock complexities are lower than the Skip List's. With the interval
$\vert I \vert = 1024$, the structure's size increases by 30\%, however both
the search complexity and the estimated processing time are twice as low as
the original model.

\begin{table}
\ra{1.1}
\centering
\resizebox{\linewidth}{!}{%
\begin{tabular}{lc@{\hs}rrrc@{\hs}rrrc@{\hs}rrc@{\whs}rrrc@{\hs}rrrc@{\hs}rr}
\toprule
$\vert I \vert$ & \phantom{a} & \multicolumn{10}{c}{Skip List} & \phantom{a} &
\multicolumn{10}{c}{SkipBlock} \\
\cmidrule{3-12} \cmidrule{14-23}
k &\phantom{a}& \multicolumn{3}{c}{60} &\phantom{a}& \multicolumn{3}{c}{6000}
&\phantom{a}& & & \phantom{a} & \multicolumn{3}{c}{60} &\phantom{a}&
\multicolumn{3}{c}{6000} & & \\
\cmidrule{3-5} \cmidrule{7-9} \cmidrule{14-16} \cmidrule{18-20}
& \phantom{a} & $T_d$ & $T_r$ & T & \phantom{a} & $T_d$ & $T_r$ & T &
\phantom{a} & SC & ZC & \phantom{a} & $T_d$ & $T_r$ & T & \phantom{a} &
$T_d$ & $T_r$ & T & \phantom{a} & SC & ZC \\
16&\phantom{a}&0.010&0.034&0.044
&\phantom{a}& 0.960 & 0.034 & 0.994 & \phantom{a}& 64 & 3998
&\phantom{a}&0.006&0.035&0.041
&\phantom{a}& 0.600 & 0.035 & 0.635 & \phantom{a}&40 & 4996 \\

32&\phantom{a}&0.019&0.032&0.051
&\phantom{a}& 1.92 & 0.032 & 1.952& \phantom{a} & 128 & 1934
&\phantom{a}&0.008&0.034&0.042
&\phantom{a}& 0.81 & 0.034 & 0.844 & \phantom{a}& 54 & 3743 \\

256&\phantom{a}&0.077&0.030&0.107
&\phantom{a}& 7.68 & 0.030 & 7.710& \phantom{a} & 512 & 234
&\phantom{a}&0.041&0.030&0.071
&\phantom{a}& 4.08 & 0.030 & 4.110& \phantom{a} & 272& 309 \\

512&\phantom{a}&0.154&0.030&0.184
&\phantom{a}& 15.36 & 0.030 & 15.390& \phantom{a} & 1024 & 117 
&\phantom{a}&0.079&0.030&0.109
&\phantom{a}& 7.89 & 0.030 & 7.920 & \phantom{a}& 526 & 229 \\

1024&\phantom{a}&0.307&0.030&0.337
&\phantom{a}&30.72 & 0.030 & 30.750& \phantom{a} & 2048 & 58
&\phantom{a}&0.155&0.030&0.185
&\phantom{a}& 15.54 & 0.030 & 15.570 & \phantom{a}& 1036 & 75 \\

\bottomrule
\end{tabular}}
\caption{Predicted processing times in seconds (n=60,000), with the
corresponding structure's search (i.e., SC) and size (i.e., ZC) costs. $T_d$
stands for the time for decoding a record, $T_r$ for the time of reading into memory a
record/synchronization point and T the sum of these two times. $k$ indicate the
number of candidates.}
\label{tab:predicted-times}
\end{table}
